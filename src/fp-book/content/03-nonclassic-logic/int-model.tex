\section{Модель интуиционистского исчисления высказываний}

В данном разделе предоставим \textit{модель} интуиционистского исчисления. Основываться она будет на трёхзначной логике, описываемой логической матрицей.

\begin{definition}
    \textit{Моделью} для $P^i$ является следующая логическая матрица:

    \begin{equation*}
        \mathbf{M} = 
        \langle
            \{0, \frac{1}{2}, 1\}, \cdot, +, \to, \sim
        \rangle
    \end{equation*}

    \begin{equation*}
        \begin{array}{ll}
            x \cdot y = \min(x,y) 
            &
            x + y = \max(x,y) 
            \\
            x \to y = \begin{cases}
                1 &\text{if}\ x \le y \\
                y &\text{else}
            \end{cases} 
            &
            \sim x = x \to 0 = \begin{cases}
                1 &\text{if}\ x = 0 \\ 
                0 &\text{else}
            \end{cases}
        \end{array}
        \\
    \end{equation*}
\end{definition}

\begin{remark}
    Легко проверить, что все аксиомы исчисления $P^i$ являются тавтологиями в описанной логической матрице.
\end{remark}

\begin{example}
    Попробуем в данной модели найти значение истинности формулы $A \lor \overline{A}$. Произведём оценку этой формулы: $f(A \lor \overline{A}) = A + \sim A$.

    \begin{tabular}{|c|c|c|c|}\hline
        $A$            & $\sim A$ & $A + \sim A$   \\\hline
        $1$            & $0$      & $1$            \\\hline
        $\sfrac{1}{2}$ & $0$      & $\sfrac{1}{2}$ \\\hline
        $0$            & $1$      & $1$            \\\hline
    \end{tabular}

    Как видно, мы получили не тавтологию, а значит формула $A \lor \overline{A}$ не выводима в $P^i$.
\end{example}

\begin{remark}
    Операция эквивалентности не вводится в исходном базисе исчисления $P^i$, однако для неё справедливо определение $A \equiv B := (A \supset B) \land (B \supset A)$. Таким образом, можно построить оценку для эквивалентности: $f(A \equiv B) = (A \to B) \cdot (B \to A)$.
\end{remark}

