\section{Связь классического и интуиционистского исчислений высказываний}

\begin{lemma}
    Если формулы $\overline{\overline{A \supset B}}$ и $\overline{\overline{A}}$ выводимы в $P^i$, то формула $\overline{\overline{B}}$ также выводима в $P^i$.
\end{lemma}

\begin{theorem}
    \textbf{(Гливенко).} Если пропозициональная формула $F$ выводима в классическом исчислении высказываний $P$ (любом), то формула $\overline{\overline{F}}$ выводима в интуиционистском исчислении высказываний $P^i$.
\end{theorem}

\begin{theorem}\label{BaseTheorem:IntuitTransform}
    Формула $A$ классического исчисления выводима тогда и только тогда, когда формула $A^*$ выводима в интуиционистском исчислении (высказываний/предикатов).

    Операция $\cdot^*$ определяется следующим образом:

    \begin{tabular}{|c|c|c|}
        \hline
        Обозначение &$A$ & $A^*$ \\
        \hline
        $\text{atom}$ &$B$ -- атом & $\overline{\overline{B}}$ \\
        $\lambda$ &$B \ \lambda\ C$ & $\overline{\overline{B^*\ \lambda\ C^*}}$ \\
        $\lnot$ & $\overline{B}$ & $\overline{B^*}$ \\
        $\kappa$ &$\kappa xB$ & $\overline{\overline{\kappa x B^*}}$ \\
        \hline
    \end{tabular}

    где $\lambda \in \{\supset, \lor, \land\}$, $\kappa \in \{\forall, \exists\}$.
\end{theorem}

\begin{remark}
    Столбец "обозначение" описывает, каким образом необходимо обозначать шаг преобразования при оформлении доказательств теорем. В силу тривиальности, шаг $\lnot$ можно опускать.
\end{remark}