\section{Модальные исчисления высказываний}

Основной особенностью модальных исчислений является прямая зависимость от их моделей. Поэтому в зависимости от свойств шкал Крипке выделяется несколько модальных пропозициональных исчислений.

\textbf{Минимальная логика $K$}

Аксиомы:
\begin{itemize}
    \item Аксиомы пропозиционального исчисления
    \item ($AK$): $\Box(A \supset B) \supset (\Box A \supset \Box B)$ (аксиома Крипке, основная модальная аксиома)
\end{itemize}

Правила вывода:
\begin{itemize}
    \item $MP$: $A, A \supset B \vdash B$
    \item $\Box$: $A \vdash \Box A$
    \item Подстановка $Sub$: $A \vdash [a/x]A$
\end{itemize}

Такой минимальной логике соответствует любая произвольная шкала Крипке. Добавление каждого из свойств регулируется следующей аксиомой:
\begin{itemize}
    \item ($R\Box$): $\Box A \supset A$ (рефлексивность)
    \item ($T\Box$): $\Box A \supset \Box \Box A$ (транизитивность)
    \item ($S\Box$): $A \supset \Box \Diamond A$ (симметричность)
\end{itemize}

Добавляя каждую такую аксиому к минимальному исчислению будем получать новое исчисление. Среди возможных вариантов выделяют следующие 3:
\begin{itemize}
    \item Исчисление $S3$: $AK + R\Box$
    \item Исчисление $S4$: $AK + R\Box + T\Box$
    \item Исчисление $S5$: $AK + R\Box + T\Box + S\Box$
\end{itemize}

\begin{remark}
    Заметим, что в определении минимального исчисления $K$ указано, что оно включает в себя аксиомы пропозиционального исчисления, но не указано, какого конкретно. По-умолчанию, будем считать, что имеется ввиду исчисление $P$. В то же время, можно взять в качестве базового исчисления интуиционистское $P^i$ и получить модальную интуиционистскую логику.

    Также, в зависимости от выбранного базового исчисления, имеют место и доказанные в этих исчислениях теоремы --- для $P$ это теорема дедукции, теорема о подстановочности эквивалентности и именованные теоремы.
\end{remark}