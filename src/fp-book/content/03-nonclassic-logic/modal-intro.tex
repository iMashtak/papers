\section{Модальная логика}

\begin{definition}
    \textit{Модальность} --- это выражение, описывающее <<оттенок истинности>>
    высказывания (уверенность, необходимость, доказуемость, осведомлённость, $\dots$)
\end{definition}

Чаще всего в суждениях используются модальности двух двойственных видов:

\begin{center}
\begin{tabular}{c c}
    Модальность необходимого & Модальность возможного \\
    необходимо & возможно \\
    обязательно & не исключено \\
    всегда & иногда \\
    должен & имеет право \\
    знает & предполагает \\
    доказуемо & непротиворечиво \\
    $\Box$ & $\Diamond$ \\
\end{tabular}
\end{center}

Таких модальностей можно предложить сколь угодно много, но есть способ 
единообразного определения смысла модальностей: в терминах модальной логики.

\begin{remark}
    $\Box$ - читается как \textit{бокс}, а $\Diamond$ как \textit{ромб}.
\end{remark}