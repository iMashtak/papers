\chapter{Теория типов}

Текст данного раздела базируется на монографии \cite{Pierce}.

\section{Чистое простое типизированное $\lambda$-исчисление}

Утверждение <<терм $t$ имеет тип $T$>> означает, что $t$ <<очевидным образом>> даёт при вычислении значение нужного вида, при этом под словами <<очевидным образом>> подразумевается, что проверить это можно статически, не производя самого вычисления $t$. Отношение типизации будем записывать следующим образом: $t: T$.

В случае объектов-значений им можно присваивать произвольные типы-значения. Например, в $\lambda$-исчисление можно погрузить булеву алгебру со значениями $true: Bool$, $false: Bool$. Функции же типизируются при помощи конструктора типов $\to$. Например, $\lambda$-абстракция $\lambda x.x$ имеет тип $T \to T$, то есть принимает значение любого типа и возвращает значение строго этого же типа. Символ $T$ в данном случае обозначает типовую переменную. Конструктор типов право-ассоциативен, то есть выражение $T_1 \to T_2 \to T_3$ обозначает $T_1 \to (T_2 \to T_3)$.

\begin{definition}
    Контекст типизации $\Gamma$ представляет собой множество переменных/термов вместе с известными для них типами.
\end{definition}

Таким образом можно записать правила типизации чистого $\lambda$- исчисления:

\begin{itemize}
    \item $\cfrac{x:T \in \Gamma}{\Gamma \vdash x:T}$
    \item $\cfrac{\Gamma,x: T_1\vdash t: T_2}{\Gamma \vdash \lambda x:T_1.t: T_1 \to T_2}$
    \item $\cfrac{\Gamma \vdash t_1: T_1 \to T_2 \ \ \ \Gamma \vdash t_2 : T_1}{\Gamma \vdash t_1t_2:T_2}$
\end{itemize}

Первое правило вывода говорит о том, что если в контексте типизации уже определён тип некоторой переменной или терма, то он остаётся в силе и в других местах использования этой переменной или терма. Второе правило вводит конструктор типов, третье правило же вводит способ сокращения типа.

Используя правила вывода в обратную сторону можно строить деревья вывода типа.

\begin{theorem}
    В любом заданном контексте типизации $\Gamma$ терм $t$ имеет не более одного типа. Более того, существует только одно дерево вывода.
\end{theorem}

\begin{theorem}
    Пусть $t$ --- замкнутый, типизированный терм (то есть $t:T$). Тогда либо $t$ является значением, либо имеется терм $t'$ такой, что $t$ возможно редуцировать до $t'$.
\end{theorem}

\begin{theorem}
    Пусть $\Gamma \vdash t:T$ и для $x$ не определён тип в контексте $\Gamma$. Тогда $\Gamma, x: S \vdash t: T$. Более того, глубина дерева вывода остаётся неизменной.
\end{theorem}

\begin{theorem}
    Если $\Gamma, x:S\vdash t: T$ и $\Gamma \vdash s: S$, то $\Gamma \vdash [s/x]t: T$. То есть тип сохраняется при подстановке переменной того же типа, что есть в терме $t$.
\end{theorem}

TODO: вывод типов комбинаторов S и K

\section{Соответствие Карри-Говарда}

Соответствие Карри-Говарда описывает в установлении взаимосвязи между множеством формул логического исчисления и системой типов. Так в случае чистого простого типизированного $\lambda$-исчисления соответствующим исчислением будет позитивное интуиционистское исчисление высказываний (позитивное --- значит без отрицания).

Соответствие состоит в следующем:

\begin{equation*}
    \begin{array}{rl}
        \text{утверждения} & \text{типы} \\
        \text{утверждение }p \supset q & \text{тип } P \to Q \\
        \text{доказательство утверждения }T & \text{терм }t \text{ типа }T \\ 
        \text{утверждение }T \text{ доказуемо} & \text{существуют термы типа }T
    \end{array}
\end{equation*}

Таким образом представляется возможным доказывать утверждения путём конструирования терма соответствующего типа.

TODO пример с комбинатором I

\section*{Задачи}\addcontentsline{toc}{section}{Задачи}

\begin{task}\label{Theorem:I-int}
    Доказать в интуиционистском исчислении высказываний ($I\lnot$): $A \supset \overline{\overline{A}}$
\end{task}
\begin{solution}
    Первым делом, проверим истинность:
    \begin{equation*}
        \begin{array}{|c|c|c|c|c|}\hline
            A            & \sim A & \sim \sim A & A \to \sim\sim A \\\hline
            1            & 0      & 1           & 1 \\\hline
            \frac{1}{2}  & 0      & 1           & 1 \\\hline
            0            & 1      & 0           & 1 \\\hline
        \end{array}
    \end{equation*}
    Доказательство существует. Имеется 2 аксиомы, работающие с отрицаниями: $RdAb$ и $K\lnot$. Их и используем:
    \begin{equation*}
    \begin{array}{llr}
        1.  & \vdash A \supset B \supset (A \supset \overline{B} \supset \overline{A})
            & RdAb
            \\
        2.  & \overline{A} \supset B \vdash \overline{A} \supset \overline{B} \supset \overline{\overline{A}} 
            & [\overline{A}/A]MP(1)
            \\
        3.  & \vdash A \supset (\overline{A} \supset B)
            & K\lnot
            \\
        4.  & A \vdash \overline{A} \supset B
            & MP(3)
            \\
        5.  & A \vdash \overline{A} \supset \overline{B}
            & [\overline{B}/B]4
            \\
        6.  & A \vdash \overline{A} \supset B \supset \overline{\overline{A}}
            & TD(MP(5,2))
            \\
        7.  & A, A \vdash \overline{\overline{A}}
            & MP(4,6)
            \\
        8.  & \vdash A \supset \overline{\overline{A}}
            & TD(7)
            \\
    \end{array}
    \end{equation*}
\end{solution}

\begin{task}
    Доказать в интуиционистском исчислении высказываний $\overline{\overline{\overline{\overline{A}} \supset \overline{\overline{B}}}} \supset \overline{\overline{\overline{A} \supset \overline{\overline{B}} \supset \overline{\overline{B}}}}$
\end{task}
\begin{solution}
    Теорема \ref{BaseTheorem:IntuitTransform} имеет вид "тогда и только тогда", это означает, что мы можем осуществлять преобразование $\cdot^*$ в обе стороны. Обозначим формулу, которую нам надо доказать как $F^*$, она принадлежит исчислению $P^i$. Тогда $F$ принадлежит исчислению $P_2$. Опишем переход к $F$:
    \begin{equation*}
    \begin{array}{llr}
        1.  & 
            \overline{\overline{
                \underbracket{\overline{\overline{A}}} 
                \supset 
                \underbracket{\overline{\overline{B}}}
            }} 
            \supset 
            \overline{\overline{
                \overline{A} 
                \supset 
                \underbracket{\overline{\overline{B}}} 
                \supset 
                \underbracket{\overline{\overline{B}}}
            }}
            & \text{atom}
            \\
        2.  & \underbracket{\overline{\overline{A \supset B}}} \supset \underbracket{\overline{\overline{\overline{A} \supset B \supset B}}}
            & \supset
            \\
        3.  & A \supset B \supset (\overline{A} \supset B \supset B)
            & \checkmark
            \\
    \end{array}
    \end{equation*}
    Получили, что $F$ --- это $A \supset B \supset (\overline{A} \supset B \supset B)$. Теперь осталось привести доказательство этой теоремы в исчислении $P_2$.
\end{solution}

\begin{task}
    Доказать: $\vdash p \supset \Diamond p$
\end{task}
\begin{solution}
    В данном выводе используется аксиома $R\Box$, значит данная теорема выполнима в исчислении $S3$, то есть на любых рефлексивных шкалах Крипке.
    \begin{equation*}
        \begin{array}{llr}
            1.  & \vdash p \supset q \supset (\overline{q} \supset \overline{p})
                & C\lnot'
                \\
            2.  & \vdash \Box \overline{p} \supset \overline{p} \supset (\overline{\overline{p}} \supset \overline{\Box \overline{p}})
                & [\Box\overline{p}/p, \overline{p}/q]1
                \\
            3.  & \vdash \Box p \supset p
                & R\Box
                \\
            4.  & \vdash \Box \overline{p} \supset \overline{p}
                & [\overline{p}/p]3
                \\
            5.  & \vdash \overline{\overline{p}} \supset \overline{\Box\overline{p}}
                & MP(4,2)
                \\
            6.  & \vdash p \supset \overline{\Box\overline{p}}
                & ES(\equiv^{\lnot}, 5)
                \\
            7.  & \vdash p \supset \Diamond p
                & Def_{\Diamond}(6)
                \\
        \end{array}
    \end{equation*}
\end{solution}

\begin{task}
    Доказать: $\vdash \Box p \supset \Diamond p$
\end{task}

\begin{task}
    Доказать: $\vdash \Box(p \land q) \equiv (\Box p \land \Box q)$
\end{task}

\begin{task}
    Доказать: $\vdash \Box p \supset \Box \Box p \equiv \Diamond \Diamond p \supset \Diamond p $
\end{task}