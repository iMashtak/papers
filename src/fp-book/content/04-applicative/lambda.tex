\section{$\lambda$-исчисление}

\begin{definition}
    Синтаксическая конструкция $\lambda x. t$ называется $\lambda$- абстракцией. Фактически это безымянная функция с одним аргументом $x$ и телом $t$.
\end{definition}

\begin{definition}
    Переменная $x$ называется связанной, если она находится в теле $t$ абстракции $\lambda x.t$. Вхождение $x$ свободно, если оно не связано никакой вышележащей абстракцией переменной $x$.
\end{definition}

\begin{definition}
    Синтаксическая конструкция $t_1\ t_2$ называется аппликацией. Смысл аппликации будет показан позже.
\end{definition}

\begin{definition}
    Пусть имеется множество переменных $V$. \textit{Множество термов} --- это наименьшее множество $T$ такое, что:
    \begin{enumerate}
        \item $\forall x \in V: x \in T$;
        \item Если $t_1 \in T$ и $x \in V$, то $\lambda x.t_1 \in T$;
        \item Если $t_1, t_2 \in T$, то $t_1\ t_2 \in T$.
    \end{enumerate}
\end{definition}

\begin{definition}
    Множество свободных переменных терма $t$ записывается как $FV(t)$ и определяется следующим образом:
    \begin{itemize}
        \item $FV(x) = \{x\}$;
        \item $FV(\lambda x. t) = FV(t) \backslash \{x\}$;
        \item $FV(t_1\ t_2) = FV(t_1) \cup FV(t_2)$
    \end{itemize}
\end{definition}

\begin{definition}
    Подстановка переменной $s$ вместо переменной $x$ в терме $t$ осуществляется следующим образом:
    \begin{itemize}
        \item $[s/x]x = s$
        \item $[s/x]y = y$, если $y \neq x$
        \item $[s/x](\lambda x.t) = \lambda x.t$
        \item $[s/x](\lambda y. t) = \lambda y.[s/x]t$, если $y \neq x$ и $y \notin FV(s)$
        \item $[s/x](t_1\ t_2)= ([s/x]t_1)([s/x]t_2)$
    \end{itemize}
\end{definition}

\begin{remark}
    Термы, отличающиеся только именами связанных переменных, взаимозаменяемы во всех контекстах.
\end{remark}

\begin{remark}
    Операция аппликации лево-ассоциативна, то есть $a\ b\ c$ эквивалентно $(a\ b)\ c$. Операция $\lambda$-абстракции же право-ассоциативна, то есть $\lambda x. \lambda y. t$ эквивалентно $\lambda x. (\lambda y. t)$.
\end{remark}

\begin{definition}
    $\alpha$-конверсией называется такое преобразование терма, при котором происходит переименование связанных переменных терма. Например, говорят, что термы $\lambda x.x$ и $\lambda y.y$ $\alpha$-эквивалентны.
\end{definition}

\begin{definition}
    Терм вида $(\lambda x. t_1)\ t_2$ называется редексом.
\end{definition}

\begin{definition}
    $\beta$-редукцией называется операция преобразования редекса $(\lambda x.t_1)\ t_2$ в терм $[t_2/x]t_1$.
\end{definition}

\begin{definition}
    Последовательное применение $\beta$-редукции по отношению к терму называют редукцией терма, либо же вычислением значения, закодированного данным термом.
\end{definition}

\begin{remark}
    Если в терме есть несколько редексов, то возникает вопрос: в каком порядке следует применять $\beta$-редукцию? Существует набор стратегий, каждую из которых далее рассмотрим.
\end{remark}

\begin{definition}
    При стратегии полной $\beta$-редукции в любой момент может сработать любой редекс. При этом редуцируются все доступные редексы.

    Например, возьмём следующий терм, содержащий несколько редексов (каждый из них подчёркнут):
    \begin{equation*}
        \begin{matrix}
            \underline{(\lambda x. \lambda y. x)((\lambda x.x)(\lambda x.x))} \\
        (\lambda x. \lambda y. x)(\underline{(\lambda x.x)(\lambda x.x)})
        \end{matrix}
    \end{equation*}

    При данной стратегии можно выбрать случайным образом один из редексов и применить $\beta$-редукцию. Итоговое выражение в любом случае будет следующим:
    \begin{equation*}
        \lambda y.\lambda x. x
    \end{equation*}
\end{definition}

\begin{definition}
    При стратегии нормального порядка вычислений всегда сначала сокращается самый левый, самый внешний редекс. При этой стратегии получим тот же результат, что и при полной $\beta$-редукции, но порядок вычислений теперь детерминирован --- сначала редуцируется редекс
    \begin{equation*}
        \underline{(\lambda x. \lambda y. x)((\lambda x.x)(\lambda x.x))}
    \end{equation*}
    а затем редекс
    \begin{equation*}
        \lambda y.(\underline{(\lambda x.x)(\lambda x.x)})
    \end{equation*}
    и получаем результат:
    \begin{equation*}
        \lambda y.\lambda x.x
    \end{equation*}
\end{definition}

\begin{definition}
    Стратегия вызова по имени не позволяет проводить редукцию внутри абстракций. Таким образом вычисление будет вестись как при стратегии нормального порядка вычислений, но остановится после редукции первого редекса.

    Именно таким образом реализуются ленивые вычисления --- аргументы функции не вычисляются, пока к ним не произошло непосредственное обращение.
\end{definition}

\begin{remark}
    Несмотря на то, что $\lambda$-абстракции являются одноместными, легко построить многоместную $\lambda$-абстракцию просто записав несколько лямбд друг за другом: $\lambda x_1. \lambda x_2. \lambda x_3. x_1\ x_2\ x_3$.
\end{remark}