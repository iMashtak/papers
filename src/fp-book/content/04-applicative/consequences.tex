\section{Построения на $\lambda$-термах и комбинаторах}

$\lambda$-исчисление, несмотря на свою кажущуюся оторванность от реального мира, напротив является одной из возможных моделей для описания оснований математики. В данном разделе опишем, как с помощью $\lambda$-термов задать логические значения, натуральные числа и операции над ними, структуры данных (пары, списки), а также рекурсию. В целом $\lambda$-исчисление само по себе в некоторой степени является языком программирования.

Для упрощения добавим в аппликативную систему объекты-значения, которые будем обозначать как $a, b, c, ...$. Это такие значения, которые не являются переменными, но по отношению к которым действуют правила подстановки.

\subsection*{Логические значения и операции над ними}

\begin{itemize}
    \item $True = \lambda x. \lambda y. x$, $True\ x\ y = x = K$
    \item $False = \lambda x. \lambda y. y$, $False\ x\ y = y$
    \item $Test = \lambda xyz.xyz$, $Test\ x\ y\ z = xyz$
\end{itemize}

\begin{equation*}
    Test\ True\ a\ b = True\ a\ b = a
\end{equation*}
\begin{equation*}
    Test\ False\ a\ b = False\ a\ b = b
\end{equation*}

\begin{itemize}
    \item $And = \lambda xy. xy\ False$
\end{itemize}

\begin{equation*}
    \begin{array}{rcl}
        And\ True\ True &=& (\lambda xy. xy\ False)\ True\ True \\ 
        &=& True\ True\ False \\
        &=& True
    \end{array}    
\end{equation*}

\begin{equation*}
    \begin{array}{rcl}
        And\ True\ False &=& (\lambda xy. xy\ False)\ True\ False \\ 
        &=& True\ False\ False \\
        &=& False
    \end{array}    
\end{equation*}

\begin{equation*}
    \begin{array}{rcl}
        And\ False\ True &=& (\lambda xy. xy\ False)\ False\ True \\ 
        &=& False\ True\ False \\
        &=& False
    \end{array}    
\end{equation*}

\subsection*{Пары}

\begin{itemize}
    \item $Pair = \lambda xyf. fxy$ --- структура, применяющая функцию $f$ по отношению к двум элементам $x$ и $y$, которые и образуют пару.
    \item $First = \lambda p.p\ True$ --- комбинатор $True$ возьмёт первый аргумент из пары $p$.
    \item $Second = \lambda p.p\ False$ --- комбинатор $False$ возьмёт второй аргумент из пары $p$.
\end{itemize}

\begin{equation*}
    \begin{array}{rcl}
        First\ (Pair\ a\ b) &=& (\lambda p.p\ True) (Pair\ a\ b) \\
        &=& Pair\ a\ b\ True \\
        &=& True\ a\ b \\ 
        &=& a
    \end{array}
\end{equation*}

\subsection*{Списки}

TODO

\subsection*{Числа}

TODO

\subsection*{Рекурсия}

Некоторые термы невозможно редуцировать до конечного состояния. Примером такого терма может служить следующий:
\begin{equation*}
    (\lambda x.xx)(\lambda x.xx)
\end{equation*}

Этот терм содержит один редекс, но при выполнении редукции получается опять тот же самый терм.

На основе этой идеи строится комбинатор неподвижной точки $Y$:
\begin{equation*}
    Y = \lambda f.(\lambda x. f(xx))(\lambda x. f(xx))
\end{equation*}

Рассмотрим применение некоторого значения к комбинатору $Y$:
\begin{equation*}
    \begin{array}{rcl}
        Ya &=& \lambda f.(\lambda x. f(xx))(\lambda x. f(xx))\ a \\
        &=& (\lambda x. a(xx)) \\
        &=& a((\lambda x. a(xx))(\lambda x. a(xx))) \\ 
        &=& a (\lambda f. (\lambda x.f(xx)(\lambda x.f(xx)))a) \\
        &=& a (Ya)
    \end{array}
\end{equation*}

Видно, что комбинатор $Y$ порождает значение $a$ бесконечное количество раз. Завершением рекурсии должен заниматься сам $a$ --- то есть $a$ должно быть функцией с условным оператором, который бы отбросил аргумент, в котором в качестве редекса выступает $Ya$. Стратегия редукции <<вызов по имени>> (ленивые вычисления) здесь наиболее уместна. Пример:
\begin{equation*}
    Y(KI) = KI(Y(KI)) = I
\end{equation*}