\section{Комбинаторная логика}

\begin{definition}
    Терм, не имеющий свободных вхождений какой-либо переменной, называется комбинатором.
\end{definition}

Примером комбинатора может служить комбинатор $I = \lambda x. x$, просто возвращающий переданное в себя значение.

Комбинаторная логика представляет собой исчисление, формулируемое следующим образом. Исходный базис:

\begin{itemize}
    \item Комбинаторы: A, B, C, ...
    \item Переменные: x, y, z, ...
\end{itemize}

Правила построения:

\begin{itemize}
    \item Если $A$ --- ппф и $B$ --- ппф, то $A\ B$ --- ппф
\end{itemize}

Аксиомы (они же правила вывода, они же правила редукции):

\begin{itemize}
    \item ($K$): $K\ x \ y = x$
    \item ($S$): $S\ x\ y\ z = x\ z\ (y\ z)$
\end{itemize}

\begin{remark}
    Названия комбинаторов такие же, как и названия аксиом классического исчисления высказываний. Совпадение?
\end{remark}

С помощью аппликации комбинаторов между собой можно определять другие комбинаторы, например, $I=SKK$. Чтобы доказать это рассмотрим терм $SKKx$ и произведём его редукцию:

\begin{equation*}
    \begin{matrix}
        1. & SKKx \\
        2. & Kx(Kx) \\
        3. & x
    \end{matrix}
\end{equation*}
