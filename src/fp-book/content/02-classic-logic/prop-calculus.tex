\section{Классическое исчисление высказываний}

Исходный базис:
\begin{itemize}
    \item Исходные символы: $($, $)$, $\supset$, $\lnot$.
    \item Переменные: $p$, $q$, $r$, $s$, \dots
\end{itemize}

Правила построения исчисления $P$:
\begin{itemize}
    \item Отдельно стоящая переменная есть ппф.
    \item Если $A$ --- ппф, то $\lnot A$ --- ппф.
    \item Если $A$ и $B$ --- ппф, то $(A \supset B)$ --- ппф.
\end{itemize}

Правила вывода:
\begin{itemize}
    \item Modus ponens ($MP$): $A, A \supset B \vdash B$
    \item Подстановка ($\beta$): $A \vdash [B/p]A$
\end{itemize}

Аксиомы:
\begin{itemize}
    \item ($K$): $p \supset (q \supset p)$ (закон утверждения консеквента)
    \item ($S$): $s \supset (p \supset q) \supset ((s \supset p) \supset (s \supset q))$ (закон самодистрибутивности импликации)
    \item ($C\lnot$): $(\lnot p \supset \lnot q) \supset (q \supset p)$ (обратный закон контрапозиции)
\end{itemize}

В дальнейшем вместо громоздкой записи $\lnot p$ будем использовать следующее обозначение: $\overline{p}$.

Операция импликации ассоциативна влево, т.е. $(p \supset q) \supset r$ эквивалентно $p \supset q \supset r$.

\begin{definition}
    {\it Доказательство} --- конечная последовательность, состоящая из нескольких ппф, из которых каждая либо является одной из аксиом, либо по правилам вывода выводится из предыдущих ппф. Таким образом доказывается последняя ппф последовательности.
\end{definition}

\begin{definition}
    Ппф называется {\it теоремой}, если она имеет доказательство.
\end{definition}

Знак $\vdash$ обозначает, что следующее после него выражение есть теорема. Если перед этим знаком стоят некоторые выражения, то это значит, что их истинность не доказана, но предполагается. В случае, если эти выражения сами являются теоремами, то можно их опустить.

В качестве производного правила вывода будем использовать операцию {\it одновременной подстановки} нескольких переменных в формулу. Доказательство того, что это возможно опустим.

\begin{remark}
    При оформлении доказательств будем использовать запись $MP(A,C)$, обозначающую применение правила вывода modus ponens, где $A=A$, $C= A\supset B$.
\end{remark}

\begin{remark}
    Будем использовать \textit{определения}. Определение вводит новый символ или выражение, которое не встречается в самой логической системе и не было ранее введено другими определениями, и объявляет его сокращением, которое будет использоваться вместо некоторой определённой формулы.

    Таким образом, дадим следующие определения:
    \begin{itemize}
        \item ($Def\lor$): $p \lor q := p \supset q \supset q$
        \item ($Def\land$): $p \land q := \lnot(\lnot p \lor \lnot q)$
        \item ($Def\ f$): $f := p \land \overline{p}$
        \item ($Def\ t$): $t := p \lor \overline{p}$
        \item ($Def\equiv$): $p \equiv q := (p \supset q) \land (q \supset p)$
        \item ($Def\subset$): $p \subset q := q \supset p$
    \end{itemize}

    При оформлении доказательств будем использовать запись $Def\cdot(A)$, обозначающую замену некоторого вхождения соответствующей подформулы $A$ по определению $Def\cdot$.
\end{remark}
