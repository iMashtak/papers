\section{Теорема о подстановочности эквивалентности}

\begin{theorem}
    (Подстановочность эквивалентности.) Если $B$ получается из $A$ подстановкой $N$ вместо одного или нескольких вхождений $M$ в $A$ (не обязательно вместо всех вхождений), если $\vdash (M \equiv N)$ и $\vdash A$, то $\vdash B$.
\end{theorem}

\begin{remark}
    В оформлении доказательств применение теоремы о подстановочности эквивалентности будем оформлять следующим образом: $ES(A, B)$. Здесь $ES$ --- это equivalence substitution, формула $A$ --- это теорема, описывающая эквивалентность двух утверждений, $B$ --- формула, в которой происходит замена вхождений подформул, описываемых теоремой $A$.
\end{remark}