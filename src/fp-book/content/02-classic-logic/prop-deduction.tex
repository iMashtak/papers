\section{Теорема дедукции}

\begin{definition}
    {\it Вариантом} ппф $A$ называется ппф, полученная из $A$ путём такой замены переменных, при которой вхождения в $A$ одной и той же переменной остаются вхождениями одной и той же переменной, а два вхождения в $A$ различных переменных остаются вхождениями различных переменных.
\end{definition}

Например, вариантами аксиомы $K$ являются $r \supset (s \supset r)$ и $q \supset (p \supset q)$, но не $p\supset(r\supset r)$ или $p\supset(p\supset p)$.

Хоть это может быть очевидно, но отметим, что
\begin{itemize}
    \item если $B$ является вариантом $A$, то $A$ является вариантом $B$;
    \item всякий вариант варианта $A$ есть вариант $A$;
    \item всякая ппф $A$ является своим собственным вариантом.
\end{itemize}

\begin{remark}
    Обозначение $A_1,...,A_n \vdash B$ называется доказательством $B$ из гипотез $A_1,...,A_n$. Часто множество гипотез называют контекстом и обозначают $\Gamma$.
\end{remark}

\begin{theorem}
    (Теорема дедукции.) Если $A_1,...,A_{n-1},A_n \vdash B$, то $A_1,...,A_{n-1} \vdash A_n \supset B$.
    % вылезло за пределы экрана
\end{theorem}
\begin{proof}
    Пусть совокупность $B_1,...,B_m$ является доказательством формулы $B$ из гипотез $A_1,...,A_n$. Таким образом, $B=B_m$. Составим конечную последовательность пп-формул $A_n \supset B_1, ..., A_n\supset B_m$, а затем покажем, как включить в эту последовательность конечное число дополнительных ппф таким образом, чтобы получающаяся последовательность оказалась доказательством формулы $A_n\supset B_m$.

    Будем рассуждать по индукции. Рассмотрим некоторую формулу $A_n \supset B_i$ и, если $i>1$, предположим, что все предшествующие вставки уже сделаны. Тогда возможны 5 случаев:
    \begin{enumerate}
        \item $B_i$ есть $A_n$. Тогда имеем $A_n \supset A_n$. Достаточно вставить доказательство теоремы $I$, из которой требуемая формула получается подстановкой.
        
        \item $B_i$ есть одно из $A_1,...,A_n$, скажем $A_r$. Тогда список формул нужно дополнить двумя: $A_r$ и $A_r \supset (A_n \supset B_i)$, что на самом деле является вариантом аксиомы $K$: $A_r \supset (A_n \supset A_r)$. По $\text{modus ponens}$ доказываем требуемую формулу $A_n \supset B_i$.
        
        \item $B_i$ есть вариант некоторой аксиомы. Тогда аналогично случаю 2 используем вариант аксиомы $K$: $B_i \supset (A_n \supset B_i)$ --- и само выражение $B_i$. Требуемая формула выводится из этих двух по $\text{modus ponens}$.
        
        \item $B_i$ выводится по $\text{modeus ponens}$ из большой посылки $B_j$ и малой посылки $B_k$, причём $j<i$ и $k<i$. Используем следующий вариант аксиомы $S$:
        \begin{equation*}
            A_n \supset (B_k \supset B_i) \supset (A_n \supset B_k \supset (A_n \supset B_i))
        \end{equation*}
        Заметим, что $B_j$ есть $B_k \supset B_i$. Тогда верно, что
        \begin{equation*}
            A_n \supset B_j \supset (A_n \supset B_k \supset (A_n \supset B_i))
        \end{equation*}
        По предположению индукции в списке формул уже имеются формулы $A_n \supset B_j$ и $A_n \supset B_k$. Значит $A_n \supset B_i$ выводится по дважды применённому $\text{modus ponens}$.
        
        \item $B_i$ является формулой, полученной в результате подстановки в некоторую $B_j$, где $j<i$. Тогда дополнительных формул вставлять не нужно, т.к. $A_n \supset B_i$ получается из $A_n \supset B_j$ той же подстановкой (с учётом имён переменных в $A_n$). \qedhere
    \end{enumerate}
\end{proof}

\begin{consequence}
    Если $A \vdash B$, то $\vdash A \supset B$.
\end{consequence}

\begin{consequence}
    Если $A_1,...,A_n \vdash B$, то $C_1,...,C_m, A_1,...,A_n \vdash B$.
\end{consequence}

\begin{consequence}
    Если $\vdash B$, то $C_1,...,C_m \vdash B$.
\end{consequence}

\begin{consequence}
    Безразлично, в каком порядке расположены гипотезы и сколько раз они повторяются.
\end{consequence}

\begin{remark}
    При оформлении доказательства будем использовать обозначение $DT(A)$, где $DT$ --- deduction theorem, а $A$ --- формула, по отношению к которой применяется теорема дедукции или одно из следствий теоремы дедукции.
\end{remark}

\begin{remark}
    Теорема дедукции позволяет передвигать символ выводимости в формуле <<влево>>. В то же время, для передвижения его <<вправо>> используется $\text{modus ponens}$. Действительно, по определению он записывается следующим образом: $A, A \supset B \vdash B$. Если же $\vdash A \supset B$, то эту гипотезу можно отбросить и сразу записать $A \vdash B$. Такую операцию над формулой при оформлении доказательства будем записывать как $MP(A)$.
\end{remark}