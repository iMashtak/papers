\section{Подстановки и замены переменных в функциональном исчислении}

В ходе доказательств нам необходимо производить различные подстановки. Так как в исчислении $F_1$ появились новые понятия: индивидные переменные, связанные, свободные переменные --- то нужно аккуратно выписать правила подстановки, связанные с ними.

\begin{itemize}
    \item Правила алфавитной замены переменной:
    \begin{enumerate}
        \item $\widetilde{[y/x]}z = z$
        \item $\widetilde{[y/x]}P(z_1, z_2, ..., z_n)=P(z_1, z_2, ...,z_n)$
        \item $\widetilde{[y/x]}\overline{A} = \overline{A}$
        \item $\widetilde{[y/x]}(A \supset B)=A \supset B$
        \item $\widetilde{[y/x]}\forall x A = \forall y [y/x]A$, если $y \notin A$
        \item $\widetilde{[y/x]}\forall y A = \forall y A$, если $y \ne x$ и $y \notin A$
    \end{enumerate}
    Алфавитная замена действует только на формулы вида $\forall x A$. То есть, мы с помощью такой замены можем менять имена связанных переменных в формулах.

    \item Правила подстановки индивидной переменной или константы $a$ вместо индивидной переменной $x$ в формулу:
    \begin{enumerate}
        \item $[a/x]x=a$
        \item $[a/x]y=y$, если $x\ne y$
        \item $[a/x]P(y_1,...,y_n)=P([a/x]y_1,...,[a/x]y_n)$
        \item $[a/x]\overline{A}=\overline{[a/x]A}$
        \item $[a/x]A\supset B=[a/x]A \supset [a/x]B$
        \item $[a/x]\forall x A=\forall x A$
        \item $[a/x]\forall y A=\forall y ([a/x]A)$, если $x\ne y \land a\ne y$
        \item $[a/x]\forall y A=\forall y' ([a/x]([y'/y]A))$, если $a=y$, причём $y'\notin FV(A)$ и $y'\ne x$
    \end{enumerate}
    С помощью такой подстановки мы можем менять имена свободных переменных в формулах.

    Замечание по последнему правилу: можно сказать, что данное правило требует провести алфавитную замену $y$ на некоторую $y' \notin FV(A)$ и только после этого проводить подстановку индивидной переменной в $A$.
\end{itemize}

\begin{example}
    Рассмотрим простой пример подстановки:

    \begin{equation*}
        [y/x,a/y](\forall x P(x,y) \supset Q(x, x, y) \land R(y))
    \end{equation*}

    По правилам подстановки индивидной переменной в формулу:

    \begin{equation*}
        \begin{array}{ll}
            1. & [y/x,a/y](\forall x P(x,y)) \supset [y/x,a/y]Q(x, x, y) \land [y/x,a/y]R(y)
            \\
            2. & \forall x P(x,a) \supset Q(y,y,a) \land R(a)
        \end{array}
    \end{equation*}

\end{example}

\begin{example}
    Рассмотрим теперь вариант, когда происходит конфликт имён переменных:

    \begin{equation*}
        [z/x, x/y, y/z](\forall x P(x) \supset \forall y \forall z Q(x,y,z))
    \end{equation*}

    \begin{equation*}
        \begin{array}{ll}
            1. & [z/x, x/y, y/z](\forall x P(x)) \supset [z/x, x/y, y/z](\forall y \forall z Q(x,y,z))
            \\
            2. & \forall x P(x) \supset \forall y' \forall z' ([z/x, x/y, y/z]Q(x,y',z'))
            \\
            3. & \forall x P(x) \supset \forall y' \forall z' Q(z, y', z')
        \end{array}
    \end{equation*}
\end{example}

В дополнение в вышесказанному, в функциональных исчислениях вводится понятие функциональных переменных. Опишем правила, по которым производится подстановка вместо них. Пусть $C$ --- некоторая произвольная формула, а $P(x_1,\dots,x_n)$ --- функциональная переменная, вместо которой мы совершаем подстановку.

\begin{enumerate}
    \item $[C/P(x_1,\dots,x_n)]P(a_1,\dots,a_n)=[a_1/x_1,\dots,a_n/x_n]C$
    \item $[C/P(x_1,\dots,x_n)]Q(a_1,\dots,a_n)=Q(a_1,\dots,a_n)$, если $P\ne Q$
    \item $[C/P(x_1,\dots,x_n)](A \supset B)=[C/P(x_1,\dots,x_n)]A \supset [C/P(x_1,\dots,x_n)]B$
    \item $[C/P(x_1,\dots,x_n)]\overline{A} = \overline{[C/P(x_1,\dots,x_n)]A}$
    \item $[C/P(x_1,\dots,x_n)]\forall x A = \forall x [C/P(x_1,\dots,x_n)]A$, если $x \notin FV(C)$
    \item $[C/P(x_1,\dots,x_n)]\forall y A = \forall y' [C/P(x_1,\dots,x_n)] [y'/y]A$, если $y \in FV(C)$, причём $y' \notin FV(C)$
\end{enumerate}

Аналогично, в последнем правиле неявно используется алфавитная замена переменной.

\begin{example}
    Рассмотрим подстановку вместо функциональной переменной:

    \begin{equation*}
        [P(x) \land R(y)/Q(x,y)](Q(x,y) \supset Q(y,x) \supset R(x))
    \end{equation*}
    \begin{equation*}
        \begin{array}{ll}
            1. & [P(x) \land R(y)/Q(x,y)]Q(x,y) \supset [P(x) \land R(y)/Q(x,y)]Q(y,x) \supset R(x)
            \\
            2. & (P(x) \land R(y)) \supset (P(y) \land R(x)) \supset R(x)
        \end{array}
    \end{equation*}

\end{example}