\section{Исчисление секвенций для предикатов}

Аналогично исчислению секвенций для высказываний можно сформулировать исчисление секвенций для предикатов. Правила вывода для логических связок работают так же, как и в исчислении $LK$, но добавляются правила вывода для кванторов.

Правила вывода (логические):
\begin{itemize}
    \item Правила введения квантора всеобщности:
    \begin{equation*}
        \frac{\Gamma,  \forall x F(x), F(t) \vdash \Delta}{\Gamma, \forall x F(x) \vdash \Delta} \tag{$\forall_L$}
    \end{equation*}
    \begin{equation*}
        \frac{\Gamma \vdash F(y),  \Delta}{\Gamma \vdash \forall x F(x), \Delta} \tag{$\forall_R$}
    \end{equation*}
    Здесь $t$ --- произвольная переменная. В то же время $c$ --- переменная, не содержащаяся в формулах из $\Gamma$, $\Delta$.

    \item Правила введения квантора существования:
    \begin{equation*}
        \frac{\Gamma,  F(y) \vdash \Delta}{\Gamma, \exists x F(x) \vdash \Delta} \tag{$\exists_L$}
    \end{equation*}
    \begin{equation*}
        \frac{\Gamma \vdash F(t), \exists x F(x), \Delta}{\Gamma \vdash \exists x F(x),  \Delta} \tag{$\exists_R$}
    \end{equation*}
    Здесь аналогично справедливы ограничения на $t$ и $y$.
\end{itemize}

\begin{remark}
    Обратный вывод при помощи секвенций также называют \textit{методом семантических таблиц Бета}.
\end{remark}

\begin{example}
    Рассмотрим ситуации, когда необходимо учитывать ограничения на переменные, описанные в правилах. Попробуем доказать формулу $\exists x P(x) \supset \forall x P(x)$. Несложно заметить, что эта формула не общезначима. Проведём вывод:

    \begin{proof}
        \begin{equation*}
            \begin{array}{c}
                \begin{array}{llr}
                    1.  & \vdash \exists x P(x) \supset \forall x P(x)
                        & \supset_R
                        \\
                    2.  & \exists x P(x) \vdash \forall x P(x)
                        & \exists_L
                        \\
                    3.  & P(c_1) \vdash \forall x P(x)
                        & \forall_R 
                        \\
                    4.  & P(c_1) \vdash P(c_2)
                        & 
                        \\
                \end{array}
                \\
            \end{array}
        \end{equation*}
    \end{proof}

    Как видно, мы, корректно применяя правила вывода, пришли к тому, что утверждение ложно. В то же время, если бы мы установили $c_1 = c_2 = c$, то "доказали" бы ложное утверждение.
\end{example}

\begin{propthm}
    $\forall x A \supset \exists x A$.
\end{propthm}
\begin{proof}
    Приведём пример успешного вывода теоремы.
    \begin{equation*}
        \begin{array}{c}
            \begin{array}{llr}
                1.  & \vdash \forall x A \supset \exists x A 
                    & \supset_R 
                    \\
                2.  &  \forall x A \vdash \exists x A
                    & \forall_L
                    \\
                3.  & \forall x A, [t/x]A \vdash \exists x A
                    & \exists_R
                    \\
                4.  & \forall x A, [t/x]A \vdash \exists x A, [t/x] A
                    & W_R
                    \\
                5.  & \forall x A, [t/x]A \vdash [t/x] A
                    & W_L
                    \\
                6.  & [t/x]A \vdash [t/x] A
                    & \checkmark
                    \\
            \end{array}
        \end{array}
    \end{equation*}
\end{proof}

\begin{example}
    Вследствие ограничений в правилах вывода для $\forall$ и $\exists$ при выводе можно попадать в "бесконечный цикл", что не позволяет нам судить об истинности предиката:
    \begin{equation*}
        \begin{array}{c}
            \begin{array}{llr}
                1.  & \vdash \forall x \exists y P(x,y) \supset \exists y \forall x P(x,y)
                    & \supset_R
                    \\
                2.  & \underline{\forall x} \exists y P(x,y) \vdash \exists y \forall x P(x,y)
                    & \forall_L
                    \\
                3.  & \forall x \exists y P(x,y), \exists y P(c_1,y)\vdash \underline{\exists y} \forall x P(x,y)
                    & \exists_R
                    \\
                4.  & \forall x \exists y P(x,y), \underline{\exists y} P(c_1,y) \vdash \exists y \forall x P(x,y), \forall x P(x,c_2)
                    & \exists_L
                    \\
                5.  & \forall x \exists y P(x,y), \exists y P(c_1,c_3)\vdash \exists y \forall x P(x,y), \underline{\forall x} P(x,c_2)
                    & \forall_R
                    \\
                6.  & \forall x \exists y P(x,y), P(c_1,c_3) \vdash \exists y \forall x P(x,y),  P(c_4,c_2)
                    & \infty
                    \\
            \end{array}
            \\
        \end{array}
    \end{equation*}
\end{example}