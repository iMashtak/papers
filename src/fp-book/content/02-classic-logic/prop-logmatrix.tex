\section{Взаимосвязь алгебры и исчисления}

\begin{definition}
    \textit{Логической матрицей} назовём набор 
    \begin{equation*}
        \mathbf{M}=\langle M, 1, \cdot, +, \to, \sim \rangle,
    \end{equation*}
    где $M$ --- непустое множество (\textit{носитель} матрицы $\mathbf{M}$), $1\in M$ (обозначает значение истины), $\sim$ --- одноместная, а $\cdot$, $+$, $\to$ --- двухместные операции на $M$, причём для любых элементов $x,y\in M$ выполняются условия:
    \begin{enumerate}
        \item если $1\to x =1$, то $x=1$;
        \item если $x\to y=y \to x = 1$, то $x=y$ 
    \end{enumerate}
\end{definition}

\begin{remark}
    Элементы множества $M$ будем называть \textit{элементами логической матрицы} $\mathbf{M}$ и иногда будем подразумевать, что $x\in M$ эквивалентно $x\in \mathbf{M}$.
\end{remark}

\begin{definition}
    \textit{Оценкой} в логической матрице $\mathbf{M}$ называется произвольная функция, которая каждой пропозициональной переменной сопоставляет некоторый элемент множества $M$.
\end{definition}

Всякую оценку $f$ можно продолжить на множество всех пропозициональных формул, положив
\begin{equation*}
    f(A\land B)=f(A)\cdot f(B);
\end{equation*}
\begin{equation*}
    f(A \lor B)= f(A)+f(B);
\end{equation*}
\begin{equation*}
    f(A\supset B)=f(A) \to f(B);
\end{equation*}
\begin{equation*}
    f(\lnot A)=\sim f(A).
\end{equation*}

Будем говорить, что формула $A$ \textit{истинна} в логической матрице $\mathbf{M}$, если $f(A)=1$, какова бы ни была оценка $f$ в $\mathbf{M}$. Если же для некоторой оценки $f$ имеет место $f(A)\ne 1$, то говорят, что формула $A$ \textit{опровергается} в матрице $\mathbf{M}$, а матрица $\mathbf{M}$ называется \textit{контрмоделью} для этой формулы.

\begin{definition}
    Логическая матрица $\mathbf{M}$ называется 
    \textit{моделью} данного пропозиционального исчисления, если все формулы, выводимые в этом исчислении, истинны в $\mathbf{M}$. Логическая матрица называется \textit{точной моделью} данного исчисления, если в этом исчислении выводимы те и только те формулы, которые истинны в этой матрице.
\end{definition}

\begin{theorem}
    Пусть пропозициональное исчисление таково, что его единственным правилом вывода является $\text{modus ponens}$. Тогда логическая матрица $\mathbf{M}$ является моделью этого исчисления, если и только если все его аксиомы истинны в $\mathbf{M}$.
\end{theorem}
\begin{proof}
    \underline{Необходимость.} Пусть логическая матрица $\mathbf{M}$ является моделью данного пропозиционального исчисления. Поскольку все аксиомы этого исчисления выводимы в нём, они истинны в $\mathbf{M}$ по определению модели.

    \underline{Достаточность.} Пусть в логической матрице $\mathbf{M}$ истинны все аксиомы пропозиционального исчисления, в котором единственным правилом вывода является $\text{modus ponens}$, и пусть $A_1,...,A_n$ --- некоторый вывод в этом исчислении. Индукцией по $i=1..n$ покажем, что формула $A_i$ истинна в матрице $\mathbf{M}$.
    \begin{itemize}
        \item При $i=1$ формула $A_1$ является аксиомой и истинна в $\mathbf{M}$ по условию.
        \item Пусть для некоторого $k<n$ каждая формула $A_i$ при $i<k$ истинна в $\mathbf{M}$. Тогда либо $A_i$ является аксиомой, а значит истинна по условию, либо получена по правилу $\text{modus ponens}$ из формул $A_l$ и $A_m$, причём $l,m<k$. Тогда $A_m$ имеет вид $A_l \supset A_k$, а значит
        \begin{equation*}
            f(A_l)=1
        \end{equation*}
        \begin{equation*}
            1=f(A_m)=f(A_l)\to f(A_k)=1 \to f(A_k)
        \end{equation*}
        Отсюда и из определения логической матрицы следует, что $f(A_k) = 1$ для любой оценки $f$, т.е. формула $A_k$ истинна в $\mathbf{M}$.
    \end{itemize}
    Таким образом, любая формула, выводимая в данном исчислении, истинна в матрице $\mathbf{M}$. Значит, $\mathbf{M}$ является моделью этого исчисления.
\end{proof}

\begin{definition}
    \textit{Моделью пропозиционального исчисления} $P$ является булева алгебра.
\end{definition}