\section{Циклические группы}

\begin{definition}
    Пусть $G$ - мультипликативная группа, $a$ - её фиксированный элемент.
    Если любой элемент $g\in G$ записывается в виде $g=a^n$ для некоторого $n\in \mathbb{Z}$, то говорят,
    что $G$ - {\it циклическая группа} с образующим $a$ (или циклическая группа, порождённая элементом $a$).
\end{definition}

Аналогично циклическая группа определяется в аддитивном случае:
$\langle a \rangle = \{na|n\in\mathbb{Z}\}$. Это, однако, не означает, что все элементы $a^n$ различны.
Условимся обозначать $(a^{-1})^k=a^{-k}$.

Простейшим примером циклической группы служит аддитивная группа натуральных чисел с нулём, порождённая $1$: $(\mathbb{N}\cup \{0\}, +)$. Или аддитивная группа отрицательных целых чисел с нулём, порождённая $-1$.

\begin{theorem}
    Любая подгруппа циклической группы есть циклическая группа.
\end{theorem}
