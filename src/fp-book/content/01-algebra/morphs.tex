\section{Морфизмы}

% Можно дать определение самого морфизма в терминах теорката, но не знаю, нужно ли.

\begin{definition}
    {\it Гомоморфизм} --- отображение алгебры на однотипную алгебру (отображение, сохраняющее главные операции).
\end{definition}

\begin{example}
    Для простоты расмотрим алгебры с одной бинарной операцией. Пусть $\mathcal{A} = (A, \circ),\ \mathcal{B} = (B, \odot),$ а $h: A \to B$ - гомоморфизм, то $\forall a, b \in A:\ h(a \circ b) = h(a) \odot h(b)$
\end{example}

\begin{definition}
    {\it Мономорфизм} --- инъективный гомоморфизм. Т.е. если $f:X\to Y$ - мономорфизм, $x, y\in X$ и $f(x)=f(y)$, то $x=y$. Или так: есть отображение из всех $x$, но не во все $y$.
\end{definition}

\begin{example}
    Например, $f: (\mathbb{Z}, +) \to (\mathbb{C}, +)$ --- $f(z)=z+iz$. Для всех элементов $z\in \mathbb{Z}$ есть соответствующий из $\mathbb{C}$ и операция сложения сохраняется. Обратно такого нет, для элемента $5+i7$ нет в $\mathbb{Z}$.
\end{example}

\begin{definition}
    {\it Эпиморфизм} --- сюръективный гомоморфизм. Т.е. если $f:X\to Y$ - эпиморфизм, то $\forall y\in Y$ $\exists x\in X:$ $y=f(x)$. Или так: есть отображение из всех $x$ во все $y$, но может быть несколько стрелок в один $y$.
\end{definition}

\begin{example}
    Любая функция вещественных чисел есть эпиморфизм. Например, $x \to x^2$.
\end{example}

\begin{definition}
    {\it Изоморфизм} --- биективный гомоморфизм. Одновременно является и моно- и эпиморфизмом.
\end{definition}

\begin{definition}
    {\it Эндоморфизм} --- гомоморфизм алгебры на саму себя.
\end{definition}

\begin{definition}
    {\it Автоморфизм} --- изоморфизм алгебры на саму себя.
\end{definition}

В качестве примера автоморфизма можно привести перестановку элементов некоторого упорядоченного множества.
