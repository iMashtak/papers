\section{Группа подстановок}

Пусть $\Omega$ - конечное множество из $n$ элементов. Природа элементов этого множества несущественна, поэтому примем, что $\Omega = \{1,2,...,n\}$.

\begin{definition}
    Элементы множества $S_n=S(\Omega)$ всех взаимно однозначных преобразований $\Omega\to\Omega$ называются \textit{подстановками}. Подстановки обозначаются греческими буквами, а единичное преобразование буквой $e$.
\end{definition}

Произвольную подстановку $\pi: i\to \pi(i)$ изображают в виде
\begin{equation*}
    \pi = 
    \begin{pmatrix}
        1 & 2 & ... & n \\
        i_1 & i_2 & ... & i_n
    \end{pmatrix}
\end{equation*}

% Здесь бы из 5го параграфа Кострикина пояснить за операцию композиции отображений. Было бы хорошо доказать, что множество, на котором определена операция композиции является группой.

Для подстановок определена операция умножения, осуществляемая в соответствии с общим правилом композиции отображений: $(\sigma\tau)(i)=\sigma(\tau(i))$. Согласно свойствам операции композиции отображений, операция композиции подстановок имеет следующие свойства:
\begin{enumerate}
    \item ассоциативность
    \item существует нейтральный элемент $e$
    \item для каждой подстановки существует обратная
\end{enumerate}

\begin{definition}
    Исходя из перечисленных свойств, множество $S_n$, рассматриваемое вместе с операцией композиции подстановок, называется \textit{симметрической группой степени $n$} (группой подстановок).
\end{definition}

Приведём пример. Пусть имеется две подстановки $\sigma, \tau \in S_n$:
\begin{equation*}
    \begin{matrix}
        \sigma=\begin{pmatrix}
            1&2&3&4\\
            2&3&4&1
        \end{pmatrix},
        &
        \tau=\begin{pmatrix}
            1&2&3&4\\
            4&3&2&1
        \end{pmatrix}
    \end{matrix}
\end{equation*}

Композиция этих подстановок вычисляется следующим образом:
\begin{equation*}
    \sigma\tau
    =
    \begin{pmatrix}
        1&2&3&4\\
        2&3&4&1
    \end{pmatrix}
    \begin{pmatrix}
        1&2&3&4\\
        4&3&2&1
    \end{pmatrix}
    =
    \begin{matrix}
        1&2&3&4\\
        4&3&2&1\\
        1&4&3&2
    \end{matrix}
    =
    \begin{pmatrix}
        1&2&3&4\\
        1&4&3&2
    \end{pmatrix}
\end{equation*}

В то же время, легко проверить, что $\sigma\tau \ne \tau\sigma$. Отсюда можем заключить, что группа подстановок не является коммутативной.

Разложим подстановки из $S_n$ в произведения более простых подстановок. Будем применять $\sigma$ саму к себе и получим следующий цикл: $1\to2\to3\to4\to1$. Кратко такую подстановку записывают в виде $\sigma=(1\;2\;3\;4)$ и называют \textit{циклом} длины $4$. Подстановка $\tau$ записывается как совокупность двух независимых циклов: $1\to4\to1$, $2\to3\to2$. Т.е. $\tau=(1\;4)(2\;3)$ является произведением двух \textit{независимых} циклов длины 2.

Заметим, что $\sigma^2=(1\;3)(2\;4)$, $\sigma^4=(\sigma^2)^2=e$, $\tau^2=e$.

\begin{definition}
    Так как $|\Omega|<\infty$, для каждой подстановки $\pi\in S_n$ найдётся однозначно определённое натуральное число $q=q(\pi)$ такое, что все различные степени содержатся во множестве $\langle \pi \rangle=\{e,\pi,\pi^2,...,\pi^{q-1}\}$, а $\pi^q=e$. Число $q$ называется \textit{порядком подстановки} $\pi$.
\end{definition}

\begin{theorem}
    Операция возведения в степень подстановки дистрибутивна относительно произведения подстановок, если рассматриваемые подстановки независимы:
    \begin{equation*}
        \forall x,y \in S_n ,\ \forall n\in \mathbb{Z}: x \cap y = \emptyset \implies (xy)^n=x^n y^n
    \end{equation*}
\end{theorem}

\begin{theorem}
    Обратным элементом для подстановки $a=(a_1 \dots a_n)$ является подстановка $a^{-1}=(a_n \dots a_1)$.
\end{theorem}
\begin{proof}
    Рассмотрим произвольный элемент $a_k$. Будем считать, что $k \ne n$ для простоты. При перемножении подстановок $a a^{-1}$ произойдёт следующая последовательность отображений: $a_k \to a_{k+1} \to a_k$. Таким образом в результирующей подстановке $\forall k: a_k \to a_k$, т.е. получили единичную подстановку $e$. Это показывает, что $a a^{-1}=e$ и доказывает, что $a^{-1}$ является обратным для $a$.
\end{proof}
