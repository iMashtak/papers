\section*{Задачи}\addcontentsline{toc}{section}{\textit{Задачи}}

\begin{task}\label{Algebra:OperationProperties}
    Определить, какими свойствами обладают следующие бинарные операции:
    \begin{enumerate}
        \item $+$ на $\mathbb{N}$.\\
        \begin{solution} 
            Ассоциативность, коммутативность. Нейтрального и обратных элементов нет. 
        \end{solution}

        \item $\times$ на $\mathbb{N}$.\\
        \begin{solution}
            Ассоциативность, коммутативность. Нейтральный элемент --- число $1$.
        \end{solution}

        \item $+$ на $\mathbb{Q}$.\\
        \begin{solution}
            Ассоциативность, коммутативность. Нейтральный элемент --- число $0$. Обратные элементы имеются за счёт отрицательных чисел.
        \end{solution}

        \item $\times$ на $\mathbb{Q}$.\\
        \begin{solution}
            Ассоциативность, коммутативность. Нейтральный элемент --- число $1$. Обратные элементы имеются за счёт дробных.
        \end{solution}

        \item $\hat{\;\;}$ (возведение в степень) на $\mathbb{N}$.\\
        \begin{solution}
            Рассмотрим ассоциативность: $a^{(b^c)} ?= (a^b)^c$. Возьмём $a=2, b=2, c=3$. Тогда $2^{(2^3)}=2^{8}=256$, $(2^2)^3=4^3=64$, $256 \ne 64$. Значит, ассоциативности нет. Нейтральный правый элемент --- $1$. Нейтрального левого элемента нет, так как $\forall n \in \mathbb{N}: 1^n=1$. Коммутативность отсутствует. Симметричный элемент подобрать не возможно --- разве что в тривиальном случае: $1^1=1$. Итого имеются свойства ассоциативности и нейтрального правого элемента.
        \end{solution}

        \item Векторное сложение на $\mathbb{V}$.\\
        \begin{solution}
            Ассоциативность, коммутативность. Нейтральный элемент --- нулевой вектор. Обратные элементы имеются отрицательных чисел.
        \end{solution}

        \item Векторное произведение на $\mathbb{V}_3$.\\
        \begin{solution}
            Не является ни ассоциативной, ни коммутативной. В то же время, не существует нейтрального элемента относительно этой операции. Достаточно удостовериться, что система уравнений, порождаемая уравнением $[a,e]=a$ с неизвестным вектором $e$ не имеет решений.
        \end{solution}
    \end{enumerate}
\end{task}

\begin{task}\label{Algebra:GrouppoidType}
    Определить класс алгебры с одной бинарной операцией.
    \begin{enumerate}
        \item $(\mathbb{Z},+)$.\\
        \begin{solution}
            \textit{Абелева группа}.
        \end{solution}

        \item $(\mathbb{Z},\times)$.\\
        \begin{solution}
            Ассоциативность, коммутативность, нейтральный элемент, но нет дробей, поэтому \textit{коммутативный моноид}.
        \end{solution}

        \item $(\mathbb{V}_n, +)$, где $\mathbb{V}_n$ --- множество векторов размерности $n$.\\
        \begin{solution}
            \textit{Абелева группа}.
        \end{solution}

        \item $(\mathbb{V}_3, \times)$, где $\mathbb{V}_3$ --- множество векторов размерности $3$.\\
        \begin{solution}
            Операция точно не ассоциативна и не коммутативна. Обратного и нейтрального элемента нет, как показано соответствующем пункте задачи \ref{Algebra:OperationProperties}. Значит, данная алгебра является \textit{группоидом}.
        \end{solution}

        \item $(\mathbb{S}, +)$, где $\mathbb{S}$ --- множество строк конечной длины в некотором заданном алфавите, а $+$ --- операция конкатенации.\\
        \begin{solution}
            Ассоциативность, нейтральный элемент. Значит, это \textit{моноид}.
        \end{solution}

        \item $(\mathbb{S}, push)$, где $\mathbb{S}$ --- множество стеков (структур данных); $push$ --- операция поэлементного добавления элементов в стек, причём если добавляемый элемент равен головному, то головной элемент удаляется из стека.\\
        \begin{solution}
            Ассоциативность отсутствует, как видно из контрпримера:
            \begin{equation*}
                ([a,b]push[c,d])push[i,j] = [d,c,a,b]push[i,j] = [j,i,d,c,a,b]
            \end{equation*}
            \begin{equation*}
                [a,b]push([c,d]push[i,j]) = [a,b]push[j,i,c,d] = [d,c,i,j,a,b]
            \end{equation*}

            Легко показать, что коммутативность отсутствует. Нейтральный элемент: $[]$, однако это только нейтральный правый, так как, например, $[]push[a,b]=[b,a]\ne [a,b]$. 
            Обозначим произвольный стек как $[A]$ и введём дополнительное обозначение: развёрнутый стек $\overline{[A]}$. Будем говорить, что $[[A],[B]]$ --- это стек, содежащий элементы стека $[A]$, а затем элементы стека $[B]$.
            Имеет место правое деление:
            \begin{equation*}
                \begin{aligned}
                    \left[ A \right]push[X]&=[B] \\
                    [\overline{[X]}, [A]]&=[B] \\
                    []push[\overline{[X]}, [A]] &= []push[B] \\
                    [\overline{[A]}, [X]] &= \overline{[B]} \\
                    [\overline{[A]}, [X]]push\overline{[A]} &= \overline{[B]}push\overline{[A]} \\
                    [X]&=[[A], \overline{[B]}]
                \end{aligned}
            \end{equation*}
            И левое деление:
            \begin{equation*}
                \begin{aligned}
                    \left[Y\right]push[A] &= [B] \\
                    [\overline{[A]}, [Y]] &= [B] \\
                    [\overline{[A]}, [Y]]push\overline{[A]} &= [B] push \overline{[A]} \\
                    [Y]&=[[A], [B]]
                \end{aligned}
            \end{equation*}
            Таким образом, получили \textit{правую лупу}.
        \end{solution}

        \item $(\mathbb{M}_{mn}, \times)$, где $\mathbb{M}_{mn}$ --- множество прямоугольных матриц данной размерности.\\
        \begin{solution}
            Операция ассоциативна. В общем случае не коммутативна. Нейтральный элемент --- единичная матрица. Обратная матрица существует только для квадратных матриц, но не для всех матриц произвольной размерности. Таким образом, получили \textit{моноид}.
        \end{solution}

        \item $(\mathbb{M}_n, \times)$, где $\mathbb{M}_n$ --- множество квадратных матриц данной размерности.\\
        \begin{solution}
            Операция ассоциативна. В общем случае не коммутативна. Нейтральный элемент --- единичная матрица. Обратная матрица существует только для невырожденных, а это не наш случай. Таким образом, снова получили \textit{моноид}.
        \end{solution}

        \item $(\mathbb{M'}_n, \times)$, где $\mathbb{M'}_n$ --- множество ортогональных матриц данной размерности.\\
        \begin{solution}
            Операция ассоциативна. В общем случае не коммутативна. Нейтральный элемент --- единичная матрица. По свойству ортогональности, обратная матрица обязательно существует, поэтому имеется обратный элемент для всех ортогональных матриц. Таким образом, получили \textit{группу}.
        \end{solution}

        \item $(\mathbb{Z}, -)$.\\
        \begin{solution}
            Рассмотрим ассоциативность: $(a-b)-c=a-(b+c)$, в то же время $a-(b-c)=a-b+c$. Таким образом, ассоциативности нет. Нейтральный правый элемент: $0$. Единственного обратного элемента нет, но имеет место правое и левое деление:
            \begin{equation*}
                \begin{cases}
                    a - x = b\\
                    y - a = b
                \end{cases}
                \implies 
                \begin{cases}
                    x = a-b \\
                    y = a+b
                \end{cases}
            \end{equation*}
            Получили \textit{правую лупу}.
        \end{solution}
    \end{enumerate}
\end{task}

\begin{task}
    Пусть $(\mathbb{M}, \circ)$ --- коммутативный моноид. Какое из перечисленных равенств является верным для любых $a,b\in \mathbb{M}$?
    \begin{enumerate}
        \item $(e \circ a)^2 \circ b \circ e = a \circ b$
        \item $(a \circ e)^2 \circ b \circ e = a \circ b \circ a$
        \item $(a \circ b \circ a^{-1})\circ b = b^2$
        \item $a^3 \circ (a \circ e^3 \circ b)^3 \circ a^2 = (a^5 \circ b^3) \circ b$
    \end{enumerate}

    \begin{solution}
        Первое выражение не подходит, так как степень при $a$ так и должна остаться равной $2$. Второе выражение верно, так как операция $\circ$ обладает свойством коммутативности. Третье выражение не верно, так как для элементов моноида не определены обратные элементы. Четвёртое выражение неверно, так как степени при $a$ и $b$ не совпадают в правой и левой частях равенства.
    \end{solution}
\end{task}

\begin{task}
    Определить класс алгебры с двумя бинарными операциями.
    \begin{enumerate}
        \item $(\mathbb{V}_3, +, \cdot)$.\\
        \begin{solution}
            Из задачи \ref{Algebra:GrouppoidType} известно, что $(\mathbb{V}_3, +)$ является абелевой группой, а $(\mathbb{V}_3, \cdot)$ --- группоидом. Значит, это \textit{кольцо}.
        \end{solution}
        \item $(\mathbb{K}_n, +, \cdot)$, где $\mathbb{K}_n$ --- множество столбцов действительных чисел высоты $n$. В качестве операций выбраны покомпонентные сложение и умножение.\\
        \begin{solution}
            Так как соответствующие алгебры $(\mathbb{R}, +)$ и $(\mathbb{R}, \cdot)$ являются абелевыми группами, то данная алгебра является \textit{полем}. Дистрибутивность операций умножения и сложения для столбцов следует из дистрибутивности умножения относительно сложения действительных чисел.
        \end{solution}
        \item $(\mathbb{M}_n, +, \times)$, где $\mathbb{M}_n$ --- множество квадратных матриц данной размерности. В качестве операций сложения и умножения выбраны обычные операции над матрицами.\\
        \begin{solution}
            Исходя из того, что $(\mathbb{M}_n, +)$ является абелевой группой, а $(\mathbb{M}_n, \times)$ моноид, получаем, что рассматриваемая алгебра является \textit{кольцом с единицей}.
        \end{solution}
        \item $(\mathbb{M}_n, +, \times)$, где $\mathbb{M}_n$ --- множество квадратных матриц данной размерности. В качестве операции сложения выбрано обычное сложение матриц, а в качестве операции умножения --- коммутатор $[A, B] = AB − BA$.\\
        \begin{solution}
            Проверим ассоциативность коммутатора:
            \begin{equation*}
                \begin{aligned}
                    \left[\left[A, B\right], C\right]=[A,B]C-C[A,B] &=(AB-BA)C-C(AB-BA)\\ &=ABC-BAC-CAB+CBA
                \end{aligned}
            \end{equation*}
            \begin{equation*}
                \begin{aligned}
                    \left[A, \left[B, C\right]\right]=A[B,C]-[B,C]A &=A(BC-CB)-(BC-CB)A\\ &=ABC-ACB-BCA+CBA
                \end{aligned}
            \end{equation*}
            Второе и третье слагаемые отличаются, поэтому ассоциативность отсутствует.

            Проверим коммутативность:
            \begin{equation*}
                [A,B]=AB-BA
            \end{equation*}
            \begin{equation*}
                [B,A]=BA-AB
            \end{equation*}
            Выражения отличаются, поэтому коммутативность отсутствует.

            Найдём нейтральный элемент:
            \begin{equation*}
                \begin{gathered}
                    \left[A, E'\right]=AE'-E'A=A\\
                    AE'=A+E'A\\
                    AE'=A(E+E')\\
                    E'=E+E'
                \end{gathered}
            \end{equation*}
            Решения такого уравнения не существует. Значит нейтрального элемента нет. Отсюда следует отсутствие обратных элементов.

            Проверим дистрибутивность коммутатора относительно сложения:
            \begin{equation*}
                [A, B+C] = A(B+C)-(B+C)A=AB-BA+AC-CA=[A,B]+[A,C]
            \end{equation*}
            \begin{equation*}
                [A+B,C]=(A+B)C-C(A+B)=AC-CA+BC-CB=[A,C]+[B,C]
            \end{equation*}
            Имеет место дистрибутивность.
            Таким образом, получили, что рассматриваемая алгебра $(\mathbb{M}_n, +, \times)$ является \textit{кольцом}.
        \end{solution}
    \end{enumerate}
\end{task}

\begin{task}
    Предоставить пример кольца, не являющегося областью целостности, т.е. имеющее делители нуля. Если записывать формально: для $(\mathbb{A}; +, -, 0; \cdot, /, 1)$ выполняется $\exists a,b\in \mathbb{A}: a \cdot b = 0$.

    \begin{solution}
        $(\mathbb{V}_3, +, \cdot)$ Известно, что произведение коллинеарных векторов равно нулевому вектору. Нулевой вектор - это нейтральный элемент относительно сложения, т.е. нуль кольца.
    \end{solution}
\end{task}

\begin{task}
    Показать, что булева алгебра логики $(\mathbb{B}=\{true, false\}, \oplus, \wedge)$ является ассоциативно-коммутативным кольцом, в котором равенство --- это эквивалентность $\equiv$, сложение --- это строгая дизъюнкция $\oplus$, умножение --- конъюнкция $\wedge$. Определить операцию вычитания, нуль и единицу кольца.
\end{task}

\begin{task}
    Пусть $a$ и $b$ — элементы некоторой коммутативной группы, такие что элемент $a$ имеет порядок 2, а у элемента $b$ порядок равен 5. Определите, чему равен порядок элемента $(ab^2a)^2$

    \begin{solution}
        \begin{equation*}
            \mathfrak{S}(ab^2a) = HOK(2, HOK(\frac{HOK(5, 2)}{2}, 2)) = 10
        \end{equation*}
        \begin{equation*}
            \mathfrak{S}((ab^2a)^2) = \frac{HOK(2, \mathfrak{S}(ab^2a))}{2} = 5
        \end{equation*}
    \end{solution}
\end{task}

\begin{task}
    Задана некоторая группа $G$ (с нейтральным элементом $e$), относительно которой известно, что порядок любого ее элемента не превосходит 4. Пусть элементы $x$ и $y$ таковы, что $x^ny^m = x$ при $n,m \in \{1,3\}$ и $x^ny^m = y$ при $n, m \in (2, 4)$. Найти возможные значения выражения: $(xy^3)^4(x^2y^3)^2$
\end{task}

\begin{task}
    Для циклической подгруппы $(\{0,3,6,9\}, +_{12})$ найти обратный элемент к элементу $9$.

    \begin{solution}
        Искомый элемент --- $3$, так как $3+_{12}9=0$.
    \end{solution}
\end{task}

\begin{task}
    $\mathbb{C}_e$ — циклическая группа с образующим элементом $e$ порядка 12.
    $\mathbb{C}_e^2$ и $\mathbb{C}_e^3$ — циклические группы с образующими элементами, соответственно, $e^2$ и $e^3$.
    Доказать либо опровергнуть утверждение: $\mathbb{C}_e^2 \cap \mathbb{C}_e^3 \prec  \mathbb{C}_e$ 
    (пересечение групп $\mathbb{C}_e^2$ и $\mathbb{C}_e^3$ есть подгруппа $\mathbb{C}_e$).

    \begin{solution}
        $\mathbb{C}_e^2 \cap \mathbb{C}_e^3 = \{e_0, e_6\}$.
        Операция замкнута, значит эта группа является подгруппой $\mathbb{C}_e$.
    \end{solution}
\end{task}

\begin{task}
    $\mathbb{C}_e$ — некоторая циклическая группа с образующим элементом $e$ и нейтральным элементом $e_0$. Элементы $\{e_0,e_2,e_4,e_6,e_8,e_{10},e_{12},e_{14}\}$ образуют подугруппу этой группы. Доказать или опровергнуть: элементы $\{e_0, e_4, e_8, e_{12}\}$ образуют подгруппу группы $\mathbb{C}_e$.
\end{task}

\begin{task}
    Найдите все подгруппы циклической группы порядка 12.

    \begin{solution}
        \begin{enumerate}
            \item $(\{ 0, 1, 2, 3, 4, 5, 6, 7, 8, 9, 10, 11\}, +_{12})$
            \item $(\{ 0, 2, 4, 6, 8, 10\}, +_{12})$
            \item $(\{ 0, 3, 6, 9\}, +_{12})$
            \item $(\{ 0, 4, 8 \}, +_{12})$
            \item $(\{ 0, 6 \}, +_{12})$
            \item $(\{ 0 \}, +_{12})$
        \end{enumerate}
    \end{solution}
\end{task}

\begin{task}
    Пусть $A$ --- 6-элементное множество, а $S(A)$ --- группа всевозможных подстановок на множестве $A$. Пусть $x=(12)(05)$, $y=(24)$, и $z=(053)(12)$. Упростить выражения:
    \begin{enumerate}
        \item $(x^2 z y^{-2})^{-1}$\\
        \begin{solution}
            $x^2=e$, $y^{-2}=(y^2)^{-1}=e^{-1}=e$, $(e z e)^{-1}=z^{-1}=(350)(21)$.\\ \textbf{Ответ:} $(350)(21)$.
        \end{solution}
        \item $(x^2 y^{-3} z)^{-1} (z x^{-2})^{-2} z^3$
    \end{enumerate}
\end{task}

\begin{task}
    Пусть $A$ --- 6-элементное множество, а $S(A)$ --- группа всевозможных подстановок на множестве $A$ с нейтральным элементом $e_0=(0)$. Упростить выражения и вычислить их значения при заданных $x$, $y$ и $z$ (использовать определение порядка элемента):
    \begin{enumerate}
        \item $z^2 (z^2 y^3)^{-2} (x y^2)^{-1}$ --- $x=(103)$, $y=(13)(04)$, $z=(15)(02)$
        
        \item $x (z x)^{-3} (z^2 y^2)^3$ --- $x=(03)$, $y=(041)(253)$, $z=(24)(05)$
        
        \item $(x^2 y^2)^3 (y z x)^{-2} (x z^3)^3$ --- $x=(031)$, $y=(240)(315)$, $z=(03)(41)$
        
        \item $(y x^3)^2 (y^2 z^4)^{-3} x^3$ --- $x=(13)(40)$, $y=(01)(42)$, $z=(351)(204)$
        
        \item $(x z)^{-5} (z y)^{-1} (y x)^{3}$ --- $x=(02)(15)$, $y=(13)(25)(40)$, $z=(23)(50)$
        
        \item $x^5 (y z^3)^{-3} (x z^4)^3$ --- $x=(23)(04)(15)$, $y=(01)(24)(35)$, $z=(34)$
        
        \item $y^2 (z y)^2 z x^2$ --- $x=(02)(15)$, $y=(135)$, $z=(024)$
        
        \item $(xyz)^{-3}$ --- $x=(12)(40)(35)$, $y=(53)(24)(10)$, $z=(43)(51)(20)$
    \end{enumerate}
\end{task}

\begin{task}
    Пусть $A$ --- 6-элементное множество, а $S(A)$ --- группа всевозможных подстановок на множестве $A$ с нейтральным элементом $e_0=(0)$. Вычислить значение выражения $(x^2 y^3)^{10}$ (выяснить, коммутируют ли элементы $x$ и $y$):
    \begin{enumerate}
        \item $x=(235)$, $y=(014)$
        \item $x=(1345)$, $y=(02)$
        \item $x=(1340)$, $y=(25)$
        \item $x=(142)$, $y=(503)$
        \item $x=(14)$, $y=(035)$
        \item $x=(24)$, $y=(135)$
        \item $x=(10)$, $y=(235)$
        \item $x=(413)$, $y=(205)$
    \end{enumerate}
\end{task}

\begin{task}
    Пусть $P$ --- алгебра патчей (\href{https://en.wikibooks.org/wiki/Understanding_Darcs/Patch_theory}{patch theory}). Исследовать свойства операций алгебры и определить её класс.
\end{task}
