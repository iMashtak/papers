\section{Бинарные операции и их свойства}

\begin{definition}
    \textit{Бинарной алгебраической операцией} на произвольном множестве $X$ называется произвольное (но фиксированное) отображение $T:X \times X \to X$.
\end{definition}

Таким образом, любой упорядоченной паре $(a,b)$ элементов $a,b\in X$ ставится в соответствие однозначно определённый элемент $T(a,b) \in X$. Часто пишут $aTb$, а ещё чаще бинарную операцию обозначают специальным символом: $+, \cdot, *$.

Условимся, что обозначение $a\cdot b$ (или $ab$) будем называть произведением, а $a+b$ --- суммой элементов $a,b \in X$.

\begin{definition}
    Бинарная операция $*$ на множестве $X$ называется \textit{ассоциативной}, если $(a*b)*c=a*(b*c)$ для всех $a,b,c \in X$.
\end{definition}

\begin{theorem}
    Если бинарная операция на $X$ ассоциативна, то результат её последовательного применения к $n$ элементам множества $X$ не зависит от расстановки скобок.
\end{theorem}
\begin{proof}
    При $n=1,2$ доказывать нечего. При $n=3$ утверждение теоремы совпадает с законом ассоциативности. Далее рассуждаем индукцией по $n$. Предположим, что для числа элементов, меньшего $n$ справедливость теоремы установлена. Тогда нужно показать, что
    \begin{equation}
        (x_1...x_k)(x_{k+1}...x_n)=(x_1...x_l)(x_{l+1}...x_n) \tag{1}
    \end{equation}
    при любых $k,l\in [1,n-1]$. Выписаны только внешние пары скобок, поскольку по предположению индукции расстановка внутренних скобок несущественна.
    Будем говорить, что $x_1x_2...x_k=(...((x_1x_2)x_3)...x_{k-1})x_k$ --- левонормированное произведение. Тогда имеем два случая:
    \begin{itemize}
        \item $k=n-1$. Тогда выражение $(1)$ примет вид $$(x_1...x_{n-1})x_n=(...(x_1x_2)...x_{n-1})x_n$$ - левонормированное произведение.
        \item $k<n-1$. Ввиду ассоциативности имеем
        \begin{align*}
            (x_1...x_k)(x_{k+1}...x_n)&=(x_1...x_k)((x_{k+1}...x_{n-1})x_n)= \\
            &= ((x_1...x_k)(x_{k+1}...x_{n-1}))x_n= \\
            &= (...((...(x_1x_2)...x_k)x_{k+1})...x_{n-1})x_n
        \end{align*}
        т.е. тоже левонормированное произведение. К такому же виду приводится и правая часть доказываемого равенства $(1)$.\qedhere
    \end{itemize}
\end{proof}
\begin{remark}
    Произведение $xx...x$ обозначают символом $x^n$, называя его $n$-й степенью элемента $x$.
\end{remark}
\begin{consequence}\label{Consequense:x^m x^n = x^{m+n}}
    С учётом введённого обозначения верно, что $x^m x^n = x^{m+n}$ и $(x^m)^n=x^{mn}$, где $m,n\in \mathbb{N}$.
\end{consequence}

\begin{definition}
    Бинарная операция $*$ на множестве $X$ называется \textit{коммутативной}, если $a*b=b*a$ для всех $a,b \in X$.
\end{definition}

\begin{remark}
    Свойства ассоциативности и коммутативности независимы.
\end{remark}

\begin{definition}
    Элемент $e_l \in X$ ($e_r \in X$) называется \textit{единичным} (или \textit{нейтральным}) слева (справа) относительно рассматриваемой операции $*$, если $\forall x\in X$ $e_l*x=x$ ($x*e_r=x$). Элемент называется \textit{нейтральным}, если он нейтрален и справа, и слева.
\end{definition}

\begin{remark}
    В общем случае может существовать произвольное количество элементов, нейтральных слева или справа.
\end{remark}

\begin{theorem}
    Если существует $e_l$ - нейтральный слева относительно $*$ и $e_r$ - нейтральный справа относительно $*$, то $e_l = e_r =e$ - нейтральный элемент, единственный для данной операции $*$.
\end{theorem}
\begin{proof}
        $$e_l = e_l * e_r = e_r = e$$
        Предположим, что $\exists f \ne e$ - нейтральный относительно $*$. Тогда $e = e * f = f$. Наше предположение ложно, значит $e$ - единственный нейтральный элемент относительно $*$.
\end{proof}

\begin{definition}
    Элемент $a\in X$ называется \textit{регулярным справа (слева)}, если из $a*b=a*c$ следует $b=c$ (т.е. $b*a=c*a \implies b=c$). Элемент называется \textit{регулярным}, если он регулярен и справа, и слева.
\end{definition}

Так, например, число $0$ не является регулярным относительно умножения на множестве чисел (наутральных, целых, вещественных, ...). Аналогично, нулевая матрица не является регулярным элементом относительно умножения матриц.

\begin{theorem}
    Если $a,b\in X$ --- регулярные элементы, а $*$ --- ассоциативная операция, то $a*b$ --- регулярный элемент.
\end{theorem}
\begin{proof}
    Запишем условие теоремы:
    \begin{equation*}
        \begin{gathered}
            a*x=a*y \implies x=y \\
            b*x=b*y \implies x=y
        \end{gathered}
    \end{equation*}
    Рассмотрим следующие выражения:
    \begin{equation*}
        \begin{gathered}
            (a*b)*x=a*(b*x)=a*(b*y)=(a*b)*y\\
            (b*a)*x=b*(a*x)=b*(a*y)=(b*a)*y
        \end{gathered}
    \end{equation*}
    \begin{equation*}
        \begin{gathered}
            a*(b*x)=a*(b*y) \implies a*x=a*y \implies x=y \\
            b*(a*x)=b*(a*y) \implies b*x=b*y \implies x=y
        \end{gathered}
    \end{equation*}
    Из записанных уравнений видно, что теорема доказана.
\end{proof}

\begin{definition}
    Элемент $a'\in X$ называется \emph{симметричным (обратным) правым (левым)} к элементу $a\in X$, если $a*a'=e$ ($a'*a=e$). Элемент $a^{-1}$ называется \textit{обратным} к $a$, если $a*a^{-1}=a^{-1}*a=e$ (симметричный и справа, и слева).
\end{definition}

\begin{theorem}
    Если $*$ ассоциативна и для элемента $a$ существует обратный элемент $a^{-1}$, то этот обратный элемент единственный.
\end{theorem}
\begin{proof}
    Предположим обратное: пусть $\exists a': a'\ne a^{-1}, a*a'=a'*a=e,$. Тогда рассмотрим следующую последовательность равенств:
    \begin{equation*}
        \begin{aligned}
            a*a' &= e\\
            a^{-1}*(a*a') &= a^{-1}*e\\
            (a^{-1}*a)*a' &= a^{-1}\\
            a' &= a^{-1}
        \end{aligned}
    \end{equation*}
    Показали, что в таком случае $a' = a^{-1}$.
\end{proof}

\begin{definition}
    Пусть $*$ --- бинарная операция на множестве $A$ и $B\subset A$. Тогда $B$ называется \textit{замкнутым} относительно операции $*$, если $\forall a,b\in B$ выполняется $a*b\in B$.
\end{definition}

\begin{definition}
    Бинарная операция умножения $\cdot$ является дистрибутивной относительно бинарной операции сложения $+$, если выполняются следующие тождества:
    \begin{itemize}
        \item $a\cdot(b+c)=(a\cdot b)+(a\cdot c)$ --- дистрибутивность слева
        \item $(a+b)\cdot c=(a\cdot c)+(b\cdot c)$ --- дистрибутивность справа
    \end{itemize}
\end{definition}
