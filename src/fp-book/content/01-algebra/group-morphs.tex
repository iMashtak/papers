\section{Морфизмы групп}

% нужно ли про таблицу Кэли?

\begin{definition}
    Две группы $(G,*)$ и $<G',\circ>$ называются \textit{изоморфными}, если существует отображение $f:G\to G'$ такое, что:
    \begin{enumerate}
        \item $f(a*b)=f(a)\circ f(b)$ для всех $a,b \in G$;
        \item $f$ биективно.
    \end{enumerate}
\end{definition}

\begin{theorem}
    Свойство 1. Единица переходит в единицу.
\end{theorem}
\begin{proof}
    Если $e$ --- единица группы $G$, то $e*a=a*e=a$ и, значит, $f(e)\circ f(a)=f(a)\circ f(e)=f(a)$, откуда следует, что $f(e)=e'$ --- единица группы $G'$. Это рассуждение верно, так как $f$ в том числе и сюрьективно, из чего следует, что элементами $f(a)$ исчерпывается вся группа $G$.
\end{proof}

\begin{theorem}
    Свойство 2. $f(a^{-1})=f(a)^{-1}$.
\end{theorem}
\begin{proof}
    Согласно свойству 1 имеем: $f(a)\circ f(a^{-1})=f(a*a^{-1})=f(e)=e'$. Отсюда
    \begin{equation*}
        f(a)^-1=f(a)^{-1}\circ e'=f(a)^{-1}\circ (f(a)\circ f(a^{-1}))=
    \end{equation*}
    \begin{equation*}
        =(f(a)^{-1}\circ f(a))\circ f(a^{-1})=e'\circ f(a^{-1}) = f(a^{-1})
    \end{equation*}
\end{proof}

\begin{theorem}
    Свойство 3. Обратное отображение $f^{-1}:G'\to G$ тоже является изоморфизмом.
\end{theorem}
\begin{proof}
    Такое отображение существует в силу биективности $f$. Теперь убедимся в выполнении первого требования для изоморфизма, т.е. покажем, что $f^{-1}(a'\circ b')=f^{-1}(a')*f^{-1}(b')$ для $a',b' \in G'$.
    В силу биективности имеем $\exists a,b \in G : f(a)=a', f(b)=b'$. Поскольку $f$ изоморфизм:
    \begin{equation*}
        a' \circ b' = f(a) \circ f(b) = f(a*b)
    \end{equation*}
    Отсюда по определению обратного отображения:
    \begin{equation*}
        f^{-1}(a'\circ b')=a*b
    \end{equation*}
    Однако $a=f^{-1}(a')$ и $b=f^{-1}(b')$, и значит:
    \begin{equation*}
        f^{-1}(a'\circ b') = f^{-1}(a') * f^{-1}(b') 
    \end{equation*}
    Что и требовалось доказать.
\end{proof}

\begin{example}
    Пусть $G_1=(\mathbb{R}_+ , \cdot)$ мультипликативная группа положительных вещественных чисел, а $G_2=(\mathbb{R}, +)$ аддитивная группа всех вещественных чисел. Тогда изоморфизмом может служить $f(x) := \ln(x)$. Действительно, $\ln ab = \ln a + \ln b$. Обратным к $f$ будет отображение $f^{-1}(x):=\exp(x)$. 
\end{example}

\begin{theorem}
    Все циклические группы одного и того же порядка (в том числе и бесконечного) изоморфны.
\end{theorem}
\begin{proof}
    Если $\langle g \rangle$ --- бесконечная циклическая группа, то все степени $g^n$ образующего $g$ различны, а значит имеется эффективный способ их перечисления и имеет место изоморфизм $f:\langle g \rangle \to (\mathbb{Z},+)$, полагая $f(g^n)=n$. Биективность очевидна, а свойство $f(g^mg^n)=f(g^n)+f(g^m)$ вытекает из теоремы \ref{Consequense:x^m x^n = x^{m+n}}.

    Рассмотрим случай конечных групп. Пусть $G=\{e,g,...,g^{q-1}\}$ и $G'=\{e',g',...,(g')^{q-1}\}$ --- две циклические группы порядка $q$. Операции не важны. Определим биективное отображение $f(g^k)=(g')^k$, $k=\overline{0,q-1}$. Положим $n+m=lq+r$, где $0\le r \le q-1$ для любых $m,n=\overline{0,q-1}$ будем иметь:
    \begin{equation*}
        f(g^{n+m})=f(g^r)=(g')^r=(g')^{n+m}=(g')^n(g')^m=f(g^n)f(g^m)
    \end{equation*}
\end{proof}

\begin{theorem}
    (Кэли.) Любая конечная группа порядка $n$ изоморфна некоторой подгруппе симметрической группы $S_n$.
\end{theorem}
\begin{proof}
    Пусть $G$ --- рассматриваемая группа, и, соответственно, $|G|=n$. Будем считать, что $S_n$ --- группа всех биективных отображений множества $G$ на себя, так как природа элементов, переставляемых элементами из $S_n$, несущественна.

    Для произвольного элемента $a\in G$ рассмотрим отображение $L_a:G\to G$, определённое формулой
    \begin{equation*}
        L_a(g)=ag
    \end{equation*}
    Если $g_1=e,...,g_n$ --- это все элементы $G$, то $a,...,ag_n$ будут теми же элементами, только расположенными в другом порядке. Это верно, так как
    \begin{equation*}
        ag_i=ag_j
    \end{equation*}
    \begin{equation*}
        a^{-1}(ag_i)=a^{-1}(ag_j)
    \end{equation*}
    \begin{equation*}
        (a^{-1}a)g_i=(a^{-1}a)g_j
    \end{equation*}
    \begin{equation*}
        g_i=g_j
    \end{equation*}

    Значит, $L_a$ --- биективное отображение (перестановка). Обратное отображение: $L_a^{-1}=L_{a^{-1}}$. Единичное отображение: $L_e$.
    В силу ассоциативности умножения имеем:
    \begin{equation*}
        L_{ab}(g)=(ab)g=a(bg)=L_a(L_b(g))
    \end{equation*}
    Таким образом, $L_{ab}=L_aL_b$.

    Множество $L_e,L_{g_2},...,L_{g_n}$ образует подгруппу (обозначим её $H$) в группе $S_n$. Имеем включение $H \subset S_n$ и имеем соответствие $L:a\to L_a$, обладающее по сказанному выше всеми свойствами изоморфизма.
\end{proof}

\begin{remark}
    Теорема Кэли выделяет универсальный объект: семейство $\{S_n|n=1,2,...\}$ симметрических групп --- вместилище всех вообще конечных групп, рассматриваемых "с точностью до изоморфизма".
\end{remark}

% может быть про гомоморфизмы, но и так уже много