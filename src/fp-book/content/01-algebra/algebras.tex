\section{Классы алгебр}

\begin{definition}
    \textit{Алгеброй} называется упорядоченная пара $\mathcal{A}=(A,\Omega)$, где $A$ --- множество элементов (\textit{носитель}), $\Omega$ --- множество операций над элементами (\textit{сигнатура}), причём все операции из множества $\Omega$ должны быть замкнуты на $A$.
\end{definition}

Часто понятие алгебры отождествляют с понятиями алгебраической структуры и алгебраической системы, однако мы будем пользоваться следующей терминологии.

\begin{definition}
    \textit{Алгебраической структурой (системой)} называется произвольное множество с заданным над ним набором операций и отношений $\mathfrak{A}=(A,F,R)$. В этом случае $A$ ---- носитель, а $\langle F,R \rangle$ --- сигнатура.
\end{definition}

Алнеброй, или общей алгеброй, или универсальной алгеброй называется раздел математики, изучающий алгебраические системы.

В качестве примера алгебры можно предложить множество целых чисел с операцией сложения: $(\mathbb{Z}, +)$. В то же время, можно ввести операцию $a\$b=a+b+ab$ и получить уже другую алгебру: $(\mathbb{Z},\$)$.

\begin{definition}
    \textit{Типом алгебры} $\mathcal{A}=(A, \{f_1,...,f_n\})$ называется список $(r(f_1),...,r(f_n))$, где $r(f_i)$ --- арность операции.
\end{definition}

\begin{definition}
    Алгебры называются \textit{однотипными}, если их типы совпадают.
\end{definition}

В алгебре в первую очередь изучаются структуры с одной бинарной операцией, поскольку это позволяет устанавливать некоторые важные свойства и классифицировать алгебраические структуры. При этом эта единственная бинарная операция может порождать дргуие операции (унарные и нуль-арные, т.е. константы), которые, формально, являются частью сигнатуры.

\begin{definition}
    \textit{Группоид (магма)} --- множество с одной бинарной операцией $\cdot:A\times A \to A$.
\end{definition}

\begin{definition}
    \textit{Правая (левая) квазигруппа} --- группоид, в котором возможно правое (левое) деление, т.е. $\forall a, b \in X\ \exists x (y): x * a = b \; (a * y = b)$. Решения этих уравнений иногда записывают так: $x = a / b (y = b / a)$  \textit{Квазигруппа} --- одновременно правая и левая квазигруппа.
\end{definition}

\begin{definition}
    \textit{Лупа} --- квазигруппа с нейтральным элементом.
\end{definition}

\begin{definition}
    \textit{Полугруппа} --- группоид, в котором умножение ассоциативно.
\end{definition}

\begin{definition}
    \textit{Моноид} --- полугруппа с нейтральным элементом.
\end{definition}

\begin{definition}
    \textit{Группа} --- моноид, в котором для каждого элемента есть обратный.
\end{definition}

\begin{definition}
    \textit{Абелева группа} --- группа, в которой операция коммутативна.
\end{definition}

Теперь рассмотрим алгебры с двумя операциями. Одну из них будем условно называть сложением, а другую --- умножением. Нейтральный элемент относительно сложения называют нулём и обозначают символом $0$, а нейтральный элемент относительно умножения называют единицей и обозначают символом $1$.

\begin{definition}
    \textit{Кольцо} --- структура с абелевой группой по сложению, в которой выполняется закон дистрибутивности сложения относительно умножения.
\end{definition}

\begin{remark}
    Выражение <<абелева группа по сложению>> (и аналогичные формулировки) понимается следующим образом: если $(A,+,\cdot)$ --- рассматриваемое кольцо, то $(A,+)$ образует абелеву группу.
\end{remark}

Кольца являются инструментом изучения структур, порождаемых двумя взаимосвязанными баинарными операциями; в данном случае эта свзь выражается наличием закона дистрибутивности. Если такой связи нет, следует рассматривать раздельно структуры, порождаемые каждой из бинарных операций по отдельности, независимо от другой.

\begin{definition}
    \textit{Ассоциативное кольцо} --- кольцо с ассоциативным умножением.
\end{definition}

\begin{definition}
    \textit{Ассоциативно-коммутативное кольцо} --- ассоциативное кольцо с коммутативным умножением.
\end{definition}

\begin{definition}
    \textit{Ассоциативное кольцо с единицей} --- ассоциативное кольцо с нейтральным элементом относительно умножения.
\end{definition}

\begin{definition}
    \textit{Область целостности} --- ассоциативно-коммутативное кольцо с единицей, в котором произведение двух ненулевых элементов не равно нулю.
\end{definition}

\begin{definition}
    \textit{Тело} --- кольцо, в котором ненулевые элементы образуют группу по умножению.
\end{definition}

\begin{definition}
    \textit{Поле} --- коммутативное кольцо, являющееся телом.
\end{definition}