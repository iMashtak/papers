\section{Кольца}

\begin{definition}
    Пусть $K$ --- непустое множество, на котором заданы две бинарные алгебраические операции: $+$ и $\cdot$ (сложение и умножение), удовлетворяющие следующим условиям:
    \begin{itemize}
        \item $(K,+)$ --- алебева группа
        \item $(K,\cdot)$ --- полугруппа
        \item умножение дистрибутивно по сложению: $(a+b)c=ac+bc$, $c(a+b)=ca+cb$ для всех $a,b,c \in K$
    \end{itemize}
    Тогда $(K,+,\cdot)$ называется {\it кольцом}.
\end{definition}

\begin{remark}
    Структура $(K,+)$ называется {\it аддитивной группой кольца}, а $(K,\cdot)$ --- его {\it мультипликативной полугруппой}. Если $(K,\cdot)$ --- моноид, то говорят, что $(K,+,\cdot)$ --- {\it кольцо с единицей}. 
\end{remark}

Существует достаточно много разновидностей колец, соответствующая теория называется общей теорией колец.

\begin{definition}
    Подмножество $L$ кольца $K$ называется {\it подкольцом}, если
    \begin{equation*}
        x,y\in L \Rightarrow x-y\in L, \;\; xy\in L
    \end{equation*}
    т.е. если $L$ --- подгруппа аддитивной группы и подполугруппа мультипликативной полугруппы кольца.
\end{definition}

Кольцо называется {\it коммутативным}, если $\forall x,y\in K:xy=yx$. В отличие от групп, коммутативное кольцо не принято называть абелевым.

\begin{example}
    $(\mathbb{Z},+,\cdot)$ --- кольцо целых чисел с обычными операциями сложения и умножения. Аналогично, кольцами с единицей являются $\mathbb{Q}$ и $\mathbb{R}$, причём включения $\mathbb{Z} \subset \mathbb{Q} \subset \mathbb{R}$ определяют цепочки подколец кольца $\mathbb{R}$.
\end{example}

\begin{example}
    Множество $m\mathbb{Z}$ целых чисел, делящихся на $m$, будет в $\mathbb{Z}$ подкольцом, правда уже без единицы при $m>1$.
\end{example}

\begin{example}
    Множество квадратных матриц порядка $n$ $M_n(\mathbb{R})$ является кольцом с единицей $1=E$. В то же время, это кольцо некоммутативное. Можно вообще рассматривать кольцо квадратных матриц $M_n(K)$ над произвольным коммутативным кольцом $K$.
\end{example}

% Есть ещё пример про кольцо функций с поточечным умножением и суммой.